% !Mode:: "TeX:UTF-8" 
\begin{conclusions}

我们的工作建立了第一个包含完整时序信息的记忆数据集,设计了基于时序信息的长期记忆模型,能够良好拟合现有的数据,并为优化间隔重复调度提供了坚实的基础。我们将最小化学习者的记忆成本作为间隔重复软件的目标,根据随机最优控制理论,我们推导出了一个有数学保证的最小化记忆成本的调度算法 SSP-MMC。SSP-MMC 将心理学上被多次验证的遗忘曲线、间隔效应等理论与现代机器学习技术相结合,降低了学习者形成长期记忆的记忆成本。相较于HLR模型,考虑时序信息的模型在拟合用户长期记忆上的精度有显著提升。并且,基于随机动态规划的间隔重复调度算法,在各项指标上都超过了之前的算法。该算法被部署在MaiMemo中以提高用户的长期记忆效率。我们在附录中提供了设计和部署的技术细节。

主要的后续工作是改进 DHP 模型,考虑用户特征对记忆状态转移的影响,并在语言学习以外的间隔重复软件中使用我们的算法来建议模型的通用性。此外,学习者使用间隔重复方法的场景十分多样化,设计匹配学习者目标的优化指标也是一个值得研究的问题。

\end{conclusions}

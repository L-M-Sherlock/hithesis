% !Mode:: "TeX:UTF-8" 
\begin{conclusions}
本文建立了第一个包含时序信息的记忆行为数据集,进行了长期记忆的回忆概率及记忆半衰期预测实验和复习调度的记忆成本约束模拟实验。通过实验,得到了长期记忆预测模型的横向对比数据和消融实验数据,以及模拟间隔重复调度的学习数据。数据分析表明,相较于HLR模型,考虑时序信息的LSTM-HLR模型在拟合用户长期记忆上的精度有显著提升,基于随机动态规划的间隔重复调度算法SSP-MMC,在各项指标上都超过了之前的算法。根据分析结果,我们验证了长期记忆模型中,时序特征对预测记忆半衰期十分有效的假设,以及间隔重复调度中,最小化记忆成本对最大化学习者记忆期望的等价性。这表明基于循环神经网络和随机最优控制理论的间隔重复调度算法能够有效地预测学习者的长期记忆状态和提高学习者的记忆效率。最终,我们达成了本项工作的目标,构建一套完整的间隔重复系统,提高学习者的记忆效率。

本文的主要创新之处归纳为以下两点:
\begin{itemize}
    \item 通过提取记忆行为序列特征,提出了一种新的基于LSTM循环神经网络的长期记忆预测模型,通过大量实验,验证了时序特征的重要性,使用3通道时序特征的LSTM-HLR模型对回忆概率预测的误差低至1.47\%,远远超过HLR模型的8.28\%;
    \item 通过应用LSTM-HLR模型模拟学习者在间隔重复过程中的长期记忆动态,搭建了一种新的间隔重复模拟环境,并引入随机最优控制的相关方法,提出了一种以最小化学习者记忆成本的间隔重复调度算法,通过模拟实验,验证了优化目标的合理性,基于随机动态规划的SSP-MMC调度算法相比基于阈值的方法节省了15.12\%的复习时间。
\end{itemize}

总体而言,本文就基于LSTM和间隔重复模型的复习调度算法做出来较为完整的探索和研究。但是,如何充分利用记忆行为数据、更丰富的特征和指标、更广泛的学科材料以及更通用的系统架构,为不同学习者在通用的学习平台和不同的学习科目中提供高效且稳定的间隔重复规划服务,仍然存在很多问题,这需要研究人员的探索。随着在线教育的普及,更多不同的学习者行为信息源也在增加,新的系统需要融合多源信息、多种行为类型和多平台数据,未来间隔重复调度算法研究仍然具有艰巨的研究挑战和广阔的应用前景。
\end{conclusions}

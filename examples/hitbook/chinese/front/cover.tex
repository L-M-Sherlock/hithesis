% !Mode:: "TeX:UTF-8"

\hitsetup{
  %******************************
  % 注意:
  %   1. 配置里面不要出现空行
  %   2. 不需要的配置信息可以删除
  %******************************
  %
  %=====
  % 秘级
  %=====
  statesecrets={公开},
  natclassifiedindex={TM301.2},
  intclassifiedindex={62-5},
  %
  %=========
  % 中文信息
  %=========
  ctitleone={基于LSTM和间隔重复},%本科生封面使用
  ctitletwo={模型的复习调度算法研究},%本科生封面使用
  ctitlecover={基于LSTM和间隔重复模型的复习调度算法研究},%放在封面中使用,自由断行
  ctitle={基于LSTM和间隔重复模型的复习调度算法研究},%放在原创性声明中使用
  csubtitle={一条副标题}, %一般情况没有,可以注释掉
  cxueke={工学},
  csubject={计算机},
  caffil={计算机科学与技术学院},
  cauthor={叶峻峣},
  csupervisor={苏敬勇教授},
  cassosupervisor={某某某教授}, % 副指导老师
  ccosupervisor={某某某教授}, % 联合指导老师
  % 日期自动使用当前时间,若需指定按如下方式修改:
  %cdate={超新星纪元},
  cstudentid={180110102},
  cstudenttype={学术学位论文}, %非全日制教育申请学位者
  cnumber={no180110102}, %编号
  cpositionname={哈铁西站}, %博士后站名称
  cfinishdate={20XX年X月---20XX年X月}, %到站日期
  csubmitdate={20XX年X月}, %出站日期
  cstartdate={3050年9月10日}, %到站日期
  cenddate={3090年10月10日}, %出站日期
  %(同等学力人员)、(工程硕士)、(工商管理硕士)、
  %(高级管理人员工商管理硕士)、(公共管理硕士)、(中职教师)、(高校教师)等
  %
  %
  %=========
  % 英文信息
  %=========
  etitle={Research on key technologies of partial porous externally pressurized gas bearing},
  esubtitle={This is the sub title},
  exueke={Engineering},
  esubject={Computer Science and Technology},
  eaffil={\emultiline[t]{School of Mechatronics Engineering \\ Mechatronics Engineering}},
  eauthor={Yu Dongmei},
  esupervisor={Professor XXX},
  eassosupervisor={XXX},
  % 日期自动生成,若需指定按如下方式修改:
  edate={December, 2017},
  estudenttype={Master of Art},
  %
  % 关键词用“英文逗号”分割
  ckeywords={间隔重复, 长期记忆, 长短时记忆网络, 随机最优控制},
  ekeywords={spaced repetition, long-term memory, long short-term memory recurrent neural network, stochastic optimal control},
}

\begin{cabstract}
间隔重复是一种记忆方法,学习者按照给定的复习周期重复记忆,可以有效地形成长期记忆。为了提高记忆效率,间隔重复的调度算法需要对学生的长期记忆进行建模,并优化复习的成本。该文提出了一种间隔重复调度算法,包含了长期记忆预测和复习最优规划两个部分。

在长期记忆预测方面,该文分析了现有的记忆模型预测误差较大的原因为缺乏对记忆行为时序特征的挖掘和利用。通过使用长短时记忆循环神经网络(LSTM)对记忆行为序列进行学习,结合半衰期回归(half-life regression, HLR)模型,相对目前最先进的模型,降低了80\%的预测误差。

在复习最优规划部分,该文对最优化的目标进行了探讨,认为最小化学习者的复习成本是一个有实际意义的指标。通过使用LSTM预测长期记忆的动态变化,该部分构建了一个马尔科夫决策过程,并将最优化问题转化为随机最短路径问题。基于随机动态规划方法,计算得到最优的复习策略。与最先进的方法相比,该策略节省了15.12\%的复习时间。
\end{cabstract}

\begin{eabstract}
  Spaced repetition is a memory method in which learners repeat their memories according to a given review cycle, which can effectively form long-term memory. To improve memory efficiency, the scheduling algorithm of spaced repetition needs to model students' long-term memory and optimize the cost of review. This paper proposes a spaced repetition scheduling algorithm that contains two parts: long-term memory prediction and optimal planning for review.

  In terms of long-term memory prediction, the paper analyzes that the reason for the large prediction error of existing memory models is the lack of mining and utilization of temporal features of memory behavior. By using long short-term memory recurrent neural network (LSTM) to learn the memory behavior sequences, combined with half-life regression (HLR) model, the prediction error is reduced by 80\% compared with the current state-of-the-art models.

  In the section on optimal planning for review, the paper explores the goal of optimization and considers minimizing the cost of review for learners as a practically meaningful metric. By using LSTM to predict the dynamics of long-term memory, the section constructs a Markov decision process and transforms the optimization problem into a stochastic shortest path problem. Based on the stochastic dynamic programming method, the optimal review strategy is computed. The strategy saves 15.12\% of the review time compared to the state-of-the-art method.
\end{eabstract}
